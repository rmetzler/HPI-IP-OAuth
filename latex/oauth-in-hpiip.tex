\section{Implementing OAuth in HPI IP}

OAuth defines three request endpoint URLs:

\begin{itemize}
\item
  Request Token URL - we use \emph{/oauth/request\_token}
\item
  User Authorization URL - we use \emph{/oauth/authorize}
\item
  Access Token URL - we use \emph{/oauth/access\_token}
\end{itemize}
These endpoints are required to be implemented with the Apache
Tapestry5 webframework that is used within the HPI IP.

\subsubsection{Request Token URL}

When the consumer sends an HTTP requests to the Request Token URL
\emph{/oauth/request\_token} there must be the following parameters
specified:

\begin{itemize}
\item
  \emph{realm}
\item
  \emph{oauth\_consumer\_key}
\item
  \emph{oauth\_signature\_method}
\item
  \emph{oauth\_callback}
\item
  \emph{oauth\_signature}
\end{itemize}
The service provider must validate the incoming OAuth message and
verify the consumer by checking his credentials against the
database. Then the service provider creates a new set of temporary
credentials and returns it in a HTTP response body using
\emph{``application/x-www-form-urlencoded''} content type.

If the request was valid the response contains the following url
encoded parameters:

\begin{itemize}
\item
  \emph{oauth\_token}
\item
  \emph{oauth\_token\_secret}
\item
  \emph{oauth\_callback\_confirmed}
\end{itemize}
This oauth token is called the \emph{request token} and it is used
to identify the OAuth authorization flow.

\subsection{User Authorization URL}

After receiving the \emph{request token} the OAuth consumer
redirects the user's web browser to the \emph{authorization url}
with the request token appended as parameter
\textbf{oauth\_token}.

The provider has to verify the identiy of the user and then ask to
authorize the requested access. Therefor the OAuth provider should
display informations about the client based on the request token
(like the OAuth client's name, url or logo) for the user to
verify.

To prevent repeated authorization attempts the provider has to
delete or mark the request token as used. If the user authorize the
consumer the user's webbrowser is redirected to the consumer's
callback url and \emph{oauth\_token} and \emph{oauth\_verifier} are
appended as parameters.

\subsection{Access Token URL}

When the user is redirected to the OAuth consumer the consumer uses
the \emph{oauth\_verifier} to obtain the permanent access token. To
do this the \emph{oauth\_verifier} is added to the list of
parameters the consumer used to obtain the request token and all
parameters are appended to the access token url.

The server must verify the validity of the request. If the request
is valid and authorized the permanent token credentials are
includes as ``application/x-www-form-urlencoded'' content type with
status code 200 (OK) in the HTTP response. The parameters are

\begin{itemize}
\item
  \emph{oauth\_token}
\item
  \emph{oauth\_token\_secret}
\end{itemize}
Once the client receives and stores the token credentials it can
use it to access protected resorces on behalf of the resource
owner. To do so the consumer has to use his credentials together
with the access token credentials received.

\subsection{Implementing OAuth}

If you have to implement an OAuth consumer you can find many client
implementations in every programming language possible.
Unfortunately the same does not hold for implementing an OAuth
provider. While there are various implementations for scripting
languages (notably Ruby, Python and PHP) we have found only one
example implementation for Java. This implementation unfortunately
isn't a simple to use framework where the developer just provides
the persistence layer and the web layer and gets the logic for
creating the tokens for free.