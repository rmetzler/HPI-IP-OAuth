\section{Introduction }
\label{sec:intro}

Every day people use many different internet websites for their online activities. While some of these websites are used by only a few to several thousend people, some other websites like webbased email programms or social networking websites are visited by millions of users on a daily basis. Google's online email application GMail, Facebook and Twitter aim to become the center of our online activities and therefor the single provider of our online identity. One way for them to achieve this is to enable third party sites to allow users to sign up and log in with accounts from the identity provider. This is possible through online identity protocols like OpenID and OAuth that are able to authenticate a user and redirect him back to the third party website.

For users this feature enables a more secure and better online experience as they are not required to register with username and password at every website they want to try out. For the third party service this often means that their sign-up conversion rate can dramatically increase which often has direct influence on their economic success.

% Outline
In this paper we describe the OAuth 1.0a protocol. Then we describe our proposed privacy service that is granted access to the HPI identity provider via OAuth. We explain the changes we made in order to enable OAuth in the HPI-IP and how the API works.

% What is ...?
% Motivation & Goals
% Outline
