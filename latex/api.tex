\subsection{Making API Requests}

Whenever the consumer tries to access the user's restricted
resource he has to add the OAuth protocol parameters to the request
by using the OAuth HTTP ``Authorization'' header field. The
required parameters are:

\begin{itemize}
\item
  \emph{oauth\_consumer\_key}
\item
  \emph{oauth\_token}
\item
  \emph{oauth\_signature\_method}
\item
  \emph{oauth\_timestamp}
\item
  \emph{oauth\_nonce}
\item
  \emph{oauth\_signature}
\end{itemize}
The server has to validate the authenticated request by
recalculating the request signature, ensuring that the combination
of \emph{nonce / timestamp / token} has never used before and
verify that the scope and status of the authorization as
represented by the OAuth request token is valid.

\subsubsection{Granularity of Rights Management}

While most existing OAuth providers (e.g. Twitter) manage rights
only at the granularity of of the access token allowing or denying
read/write access to every of the user's resources, it is often
feasible to manage rights with finer granularity. In fact this is
what we need to do in the HPI IP in order to allow users to grant
access to some relevant attributes of their identities and deny
access to others.

By this means that not only the API consumers use is becoming more
complex, the complexity for the user interface to manage access
rights for consumers transparently increases too. If this problem
isn't solved in the user interface it could lead to conflicts over
privacy issues like it is the case with Facebook's third party
apps.

In order to have maximum fine granularity of rights management we
decided to manage rights to every of attributes of each digital
identity seperatly.

\subsubsection{Accessing the User's Restricted Resources}

In today's internet world there are many different technologies for
providing API access but RESTful Web Services prevail. We decided
to build the neccessary API in a restful way using URLs to name
resources, HTTP verbs for methods and HTTP response codes to signal
results. [REST] This way we have to decide which resources should
be available and how they are represented.

The OAuth access token is essentially the right to access some of
the associated user's restricted resources. The OAuth client has to
query the API to find out which specific resources it can access in
behalf of the user. Therefor we need one URL endpoint that is the
same for every user. By querying it with the access token the
consumer specifies the user and gets some more information about
which resources are accessable with the provided token.

\begin{itemize}
\item
  the resource of the user who granted the access token should be
  \emph{HPIIP/api/user}
\end{itemize}
The representation of this resource in JSON should be something
like this, showing only the values that are available to the
requesting consumer:

\begin{verbatim}
{
    "user" : {
        "username" : "richard.metzler",
        "identities" : [
            {
                "main identity" : [
                    {
                        "id" : "123456",
                        "attribute" : "email",
                        "value" : "richard@...",
                        "resource" : "HPIIP/api/attribute/123456",
                        "writable" : true
                    }, ...
                ]

            }, ...
        ]
    }
}
\end{verbatim}
As you can see, the 3rd party service is able to find out the
resource for reading and updating the email resource. Reading is
done with an HTTP GET request, while updating is by sending an HTTP
POST request to the \emph{HPIIP/api/attribute/\{id\}} resource. The
resource is again represented in JSON format:

\begin{verbatim}
{
    "id" : "123456",
    "attribute" : "email",
    "value" : "richard@...",
    "resource" : "HPIIP/api/attribute/123456",
    "writable" : true
}
\end{verbatim}
If the client updates the value the attribute only the attribute
\emph{value} needs to be in representation.

Whenever an OAuth consumer wants to access a resouce the OAuth
provider has to verify if the requested operation is granted to
provided access token. It has to be denied otherwise using the HTTP
response code \textbf{401 (``Unauthorized'')}.

\subsubsection{Implementation}

The proposed implementation of the described API is to use
\emph{Jersey}, Sun's implementation of the JAX-RS specification.
JAX-RS describes a Java API to build RESTful webservices by using
Java5 style annotations. These webservices have to run in a servlet
container like Apache Tomcat or Jetty.

Because the existing HPI IP uses the Tapestry5 webframework we had
to decide if the API and the existing application should be
deployed next to each other and use their own database bindings or
deploy it as one application in the same directory. Because we
wanted to reuse the existing database bindings we decided to use
the Tapestry-Jersey integration provided by Blue Tang
Studio.[1][2]
